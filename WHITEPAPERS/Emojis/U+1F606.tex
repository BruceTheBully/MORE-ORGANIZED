\documentclass{article}
\usepackage{amsmath}
\usepackage{amssymb}
\begin{document}

\section*{Formal Semantic Proof of Emoji 😆 (Unicode U+1F606)}

\paragraph{Setup: Semantic Morphology Framework}

Let $\mathcal{E}$ be the set of all emojis.

Let $\sigma : \mathcal{E} \to \mathcal{S}$ be a semantic mapping from emojis to their semantic vectors in a conceptual semantic space $\mathcal{S}$.

Let $\mathrm{Laughter} \in \mathcal{S}$ be the canonical vector representing ``laughter'' or ``amusement.''

Let $\mathrm{Face}_{\mathrm{laugh}} \subset \mathcal{E}$ be the subset of emojis representing laughing faces.

Let $U$ be the universal semantic space for human emotion and expression.

\paragraph{Step 1: Identify emoji class}

\[
\text{😆} \in \mathrm{Face}_{\mathrm{laugh}} \subset \mathcal{E}
\]

\paragraph{Step 2: Define semantic vector for 😆}

The semantic mapping assigns

\[
\sigma(\text{😆}) = v \in U
\]

where $v$ encodes facial expression features:  
\begin{itemize}
    \item Mouth shape $m = $ wide open, showing teeth (laughing)
    \item Eye shape $e = $ tightly closed (intense laughter)
\end{itemize}

\paragraph{Step 3: Semantic composition via Morphological Calculus}

Represent the emoji's semantics as

\[
\sigma(\text{😆}) = \phi_m(m) \oplus \phi_e(e) \oplus \phi_c(c)
\]

where:  
\begin{itemize}
    \item $\phi_m$ maps mouth shape to semantic features,
    \item $\phi_e$ maps eye shape,
    \item $\phi_c$ maps contextual cues (e.g., intensity of eyes, mouth openness),
    \item $\oplus$ is the semantic composition operator.
\end{itemize}

\paragraph{Step 4: Semantic feature entailments}

\[
\begin{cases}
\phi_m(m) \to +\mathrm{Valence} \\
\phi_e(e) \to +\mathrm{Arousal}^{\uparrow} \\
\phi_c(c) \to +\mathrm{Amusement}
\end{cases}
\implies
\sigma(\text{😆}) = +\mathrm{Valence} \wedge +\mathrm{Arousal}^{\uparrow} \wedge +\mathrm{Amusement}
\]

which matches the vector representation of $\mathrm{Laughter} \in U$.

\paragraph{Step 5: Morphic equivalence to Laughter}

Define $\mathrm{Laughter} \in U$ as a target semantic vector with these features:

\[
\mathrm{Laughter} = +\mathrm{Valence} \wedge +\mathrm{HighArousal} \wedge +\mathrm{Amusement}
\]

Then

\[
\sigma(\text{😆}) \equiv \mathrm{Laughter}
\]

\paragraph{Step 6: Formal semantic weight proof}

Since $\sigma(\text{😆})$ equals the semantic vector $\mathrm{Laughter}$, and $\mathrm{Laughter}$ is a primitive concept in $U$,  
and $\text{😆}$ is morphologically constructed by combining facial feature morphs with positive valence, increased arousal, and amusement,  
we conclude

\[
\boxed{
\mathrm{SemanticWeight}(\text{😆}) = \mathrm{Laughter}
}
\]

\end{document}
