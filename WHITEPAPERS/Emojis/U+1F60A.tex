\documentclass{article}
\usepackage{amsmath}
\usepackage{amssymb}
\begin{document}

\section*{Formal Semantic Proof of Emoji 😊 (Unicode U+1F60A)}

\paragraph{Setup: Semantic Morphology Framework}

Let $\mathcal{E}$ be the set of all emojis.

Let $\sigma : \mathcal{E} \to \mathcal{S}$ be a semantic mapping from emojis to their semantic vectors in a conceptual semantic space $\mathcal{S}$.

Let $\mathrm{Blush} \in \mathcal{S}$ be the canonical vector representing ``shyness,'' ``warmth,'' or ``gentle happiness.''

Let $\mathrm{Face}_{\mathrm{blush}} \subset \mathcal{E}$ be the subset of emojis representing blushing or gentle smiling faces.

Let $U$ be the universal semantic space for human emotion and expression.

\paragraph{Step 1: Identify emoji class}

\[
\text{😊} \in \mathrm{Face}_{\mathrm{blush}} \subset \mathcal{E}
\]

\paragraph{Step 2: Define semantic vector for 😊}

The semantic mapping assigns

\[
\sigma(\text{😊}) = v \in U
\]

where $v$ encodes facial expression features:  
\begin{itemize}
    \item Mouth shape $m = $ gentle closed smile
    \item Eye shape $e = $ smiling, relaxed eyes
    \item Cheek color $c = $ subtle blush (reddish tint)
\end{itemize}

\paragraph{Step 3: Semantic composition via Morphological Calculus}

Represent the emoji's semantics as

\[
\sigma(\text{😊}) = \phi_m(m) \oplus \phi_e(e) \oplus \phi_b(c)
\]

where:  
\begin{itemize}
    \item $\phi_m$ maps mouth shape to semantic features,
    \item $\phi_e$ maps eye shape,
    \item $\phi_b$ maps blush coloration (cheek tint) to semantic warmth,
    \item $\oplus$ is the semantic composition operator.
\end{itemize}

\paragraph{Step 4: Semantic feature entailments}

\[
\begin{cases}
\phi_m(m) \to +\mathrm{Valence} \\
\phi_e(e) \to +\mathrm{Calm} \\
\phi_b(c) \to +\mathrm{Warmth} \wedge +\mathrm{Shyness}
\end{cases}
\implies
\sigma(\text{😊}) = +\mathrm{Valence} \wedge +\mathrm{Calm} \wedge +\mathrm{Warmth} \wedge +\mathrm{Shyness}
\]

which matches the vector representation of $\mathrm{Blush} \in U$.

\paragraph{Step 5: Morphic equivalence to Blush}

Define $\mathrm{Blush} \in U$ as a target semantic vector with these features:

\[
\mathrm{Blush} = +\mathrm{Valence} \wedge +\mathrm{Calm} \wedge +\mathrm{Warmth} \wedge +\mathrm{Shyness}
\]

Then

\[
\sigma(\text{😊}) \equiv \mathrm{Blush}
\]

\paragraph{Step 6: Formal semantic weight proof}

Since $\sigma(\text{😊})$ equals the semantic vector $\mathrm{Blush}$, and $\mathrm{Blush}$ is a primitive concept in $U$,  
and $\text{😊}$ is morphologically constructed by combining facial feature morphs with positive valence, calmness, warmth, and shyness,  
we conclude

\[
\boxed{
\mathrm{SemanticWeight}(\text{😊}) = \mathrm{Blush}
}
\]

\end{document}
